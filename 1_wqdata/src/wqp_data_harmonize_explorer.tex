\documentclass[]{article}
\usepackage{lmodern}
\usepackage{amssymb,amsmath}
\usepackage{ifxetex,ifluatex}
\usepackage{fixltx2e} % provides \textsubscript
\ifnum 0\ifxetex 1\fi\ifluatex 1\fi=0 % if pdftex
  \usepackage[T1]{fontenc}
  \usepackage[utf8]{inputenc}
\else % if luatex or xelatex
  \ifxetex
    \usepackage{mathspec}
  \else
    \usepackage{fontspec}
  \fi
  \defaultfontfeatures{Ligatures=TeX,Scale=MatchLowercase}
\fi
% use upquote if available, for straight quotes in verbatim environments
\IfFileExists{upquote.sty}{\usepackage{upquote}}{}
% use microtype if available
\IfFileExists{microtype.sty}{%
\usepackage{microtype}
\UseMicrotypeSet[protrusion]{basicmath} % disable protrusion for tt fonts
}{}
\usepackage[margin=1in]{geometry}
\usepackage{hyperref}
\hypersetup{unicode=true,
            pdftitle={WQP Parameter and Method exploration},
            pdfborder={0 0 0},
            breaklinks=true}
\urlstyle{same}  % don't use monospace font for urls
\usepackage{color}
\usepackage{fancyvrb}
\newcommand{\VerbBar}{|}
\newcommand{\VERB}{\Verb[commandchars=\\\{\}]}
\DefineVerbatimEnvironment{Highlighting}{Verbatim}{commandchars=\\\{\}}
% Add ',fontsize=\small' for more characters per line
\usepackage{framed}
\definecolor{shadecolor}{RGB}{248,248,248}
\newenvironment{Shaded}{\begin{snugshade}}{\end{snugshade}}
\newcommand{\AlertTok}[1]{\textcolor[rgb]{0.94,0.16,0.16}{#1}}
\newcommand{\AnnotationTok}[1]{\textcolor[rgb]{0.56,0.35,0.01}{\textbf{\textit{#1}}}}
\newcommand{\AttributeTok}[1]{\textcolor[rgb]{0.77,0.63,0.00}{#1}}
\newcommand{\BaseNTok}[1]{\textcolor[rgb]{0.00,0.00,0.81}{#1}}
\newcommand{\BuiltInTok}[1]{#1}
\newcommand{\CharTok}[1]{\textcolor[rgb]{0.31,0.60,0.02}{#1}}
\newcommand{\CommentTok}[1]{\textcolor[rgb]{0.56,0.35,0.01}{\textit{#1}}}
\newcommand{\CommentVarTok}[1]{\textcolor[rgb]{0.56,0.35,0.01}{\textbf{\textit{#1}}}}
\newcommand{\ConstantTok}[1]{\textcolor[rgb]{0.00,0.00,0.00}{#1}}
\newcommand{\ControlFlowTok}[1]{\textcolor[rgb]{0.13,0.29,0.53}{\textbf{#1}}}
\newcommand{\DataTypeTok}[1]{\textcolor[rgb]{0.13,0.29,0.53}{#1}}
\newcommand{\DecValTok}[1]{\textcolor[rgb]{0.00,0.00,0.81}{#1}}
\newcommand{\DocumentationTok}[1]{\textcolor[rgb]{0.56,0.35,0.01}{\textbf{\textit{#1}}}}
\newcommand{\ErrorTok}[1]{\textcolor[rgb]{0.64,0.00,0.00}{\textbf{#1}}}
\newcommand{\ExtensionTok}[1]{#1}
\newcommand{\FloatTok}[1]{\textcolor[rgb]{0.00,0.00,0.81}{#1}}
\newcommand{\FunctionTok}[1]{\textcolor[rgb]{0.00,0.00,0.00}{#1}}
\newcommand{\ImportTok}[1]{#1}
\newcommand{\InformationTok}[1]{\textcolor[rgb]{0.56,0.35,0.01}{\textbf{\textit{#1}}}}
\newcommand{\KeywordTok}[1]{\textcolor[rgb]{0.13,0.29,0.53}{\textbf{#1}}}
\newcommand{\NormalTok}[1]{#1}
\newcommand{\OperatorTok}[1]{\textcolor[rgb]{0.81,0.36,0.00}{\textbf{#1}}}
\newcommand{\OtherTok}[1]{\textcolor[rgb]{0.56,0.35,0.01}{#1}}
\newcommand{\PreprocessorTok}[1]{\textcolor[rgb]{0.56,0.35,0.01}{\textit{#1}}}
\newcommand{\RegionMarkerTok}[1]{#1}
\newcommand{\SpecialCharTok}[1]{\textcolor[rgb]{0.00,0.00,0.00}{#1}}
\newcommand{\SpecialStringTok}[1]{\textcolor[rgb]{0.31,0.60,0.02}{#1}}
\newcommand{\StringTok}[1]{\textcolor[rgb]{0.31,0.60,0.02}{#1}}
\newcommand{\VariableTok}[1]{\textcolor[rgb]{0.00,0.00,0.00}{#1}}
\newcommand{\VerbatimStringTok}[1]{\textcolor[rgb]{0.31,0.60,0.02}{#1}}
\newcommand{\WarningTok}[1]{\textcolor[rgb]{0.56,0.35,0.01}{\textbf{\textit{#1}}}}
\usepackage{longtable,booktabs}
\usepackage{graphicx,grffile}
\makeatletter
\def\maxwidth{\ifdim\Gin@nat@width>\linewidth\linewidth\else\Gin@nat@width\fi}
\def\maxheight{\ifdim\Gin@nat@height>\textheight\textheight\else\Gin@nat@height\fi}
\makeatother
% Scale images if necessary, so that they will not overflow the page
% margins by default, and it is still possible to overwrite the defaults
% using explicit options in \includegraphics[width, height, ...]{}
\setkeys{Gin}{width=\maxwidth,height=\maxheight,keepaspectratio}
\IfFileExists{parskip.sty}{%
\usepackage{parskip}
}{% else
\setlength{\parindent}{0pt}
\setlength{\parskip}{6pt plus 2pt minus 1pt}
}
\setlength{\emergencystretch}{3em}  % prevent overfull lines
\providecommand{\tightlist}{%
  \setlength{\itemsep}{0pt}\setlength{\parskip}{0pt}}
\setcounter{secnumdepth}{0}
% Redefines (sub)paragraphs to behave more like sections
\ifx\paragraph\undefined\else
\let\oldparagraph\paragraph
\renewcommand{\paragraph}[1]{\oldparagraph{#1}\mbox{}}
\fi
\ifx\subparagraph\undefined\else
\let\oldsubparagraph\subparagraph
\renewcommand{\subparagraph}[1]{\oldsubparagraph{#1}\mbox{}}
\fi

%%% Use protect on footnotes to avoid problems with footnotes in titles
\let\rmarkdownfootnote\footnote%
\def\footnote{\protect\rmarkdownfootnote}

%%% Change title format to be more compact
\usepackage{titling}

% Create subtitle command for use in maketitle
\newcommand{\subtitle}[1]{
  \posttitle{
    \begin{center}\large#1\end{center}
    }
}

\setlength{\droptitle}{-2em}
  \title{WQP Parameter and Method exploration}
  \pretitle{\vspace{\droptitle}\centering\huge}
  \posttitle{\par}
  \author{}
  \preauthor{}\postauthor{}
  \date{}
  \predate{}\postdate{}

\usepackage{booktabs}
\usepackage{longtable}
\usepackage{array}
\usepackage{multirow}
\usepackage[table]{xcolor}
\usepackage{wrapfig}
\usepackage{float}
\usepackage{colortbl}
\usepackage{pdflscape}
\usepackage{tabu}
\usepackage{threeparttable}
\usepackage[normalem]{ulem}

\begin{document}
\maketitle

\hypertarget{harmonizing-disparate-data}{%
\section{Harmonizing disparate data}\label{harmonizing-disparate-data}}

The data from the water quality portal includes a wide range of methods
and characteristic names. For example in the ``chlorophyll'' this can be
chlorophyll a, b, or both and retrieved using a variety of methods. To
know which methods and characteristic names to keep and use, we must
first get a better understanding of the type of data we have.

Here we are harmonizing the entirety of the water quality portal data
even though the vast majority of these sites will not be landsat
visible. The computation time to do it for a few extra million samples
is not onerous and the intermediate mostly harmonized full dataset will
likely be useful for other uses.

We'll start with the easiest first. Secchi depth

\hypertarget{secchi-depth}{%
\subsection{Secchi depth}\label{secchi-depth}}

\hypertarget{secchi-table}{%
\subsubsection{Secchi Table}\label{secchi-table}}

In many ways, the secchi disk depth measurement is the easiest water
quality parameter to harmonize, because there is really only one method
for measuring secchi disk depth (it's in the name after all), and there
should always be units of depth (m, ft, inches, cm, etc\ldots{}). So to
harmonize secchi depth measurements we simpy drop all units that are not
units of depth and convert all units to a single kind with a lookup
table.

\begin{Shaded}
\begin{Highlighting}[]
\CommentTok{#Read in the raw data from '1_wqdata/out'}
\NormalTok{secchi <-}\StringTok{ }\KeywordTok{read_feather}\NormalTok{(}\StringTok{'1_wqdata/tmp/wqp/all_raw_secchi.feather'}\NormalTok{) }\OperatorTok
\StringTok{  }\KeywordTok{wqp.renamer}\NormalTok{() }
\end{Highlighting}
\end{Shaded}

\begin{verbatim}
## [1] "You dropped 1894 samples because the sample medium was not labeled as Water"
\end{verbatim}

\begin{Shaded}
\begin{Highlighting}[]
\CommentTok{#Summarize by characteristic name and unit code and print}
\NormalTok{secchi }\OperatorTok
\StringTok{  }\KeywordTok{group_by}\NormalTok{(parameter,units) }\OperatorTok
\StringTok{  }\KeywordTok{summarize}\NormalTok{(}\DataTypeTok{count=}\KeywordTok{n}\NormalTok{()) }\OperatorTok
\StringTok{  }\KeywordTok{arrange}\NormalTok{(}\KeywordTok{desc}\NormalTok{(count)) }\OperatorTok
\StringTok{  }\KeywordTok{kable}\NormalTok{(.,}\StringTok{'html'}\NormalTok{,}\DataTypeTok{caption=}\StringTok{'All secchi parameter and unit combinations'}\NormalTok{) }\OperatorTok
\StringTok{  }\KeywordTok{kable_styling}\NormalTok{() }\OperatorTok
\StringTok{  }\KeywordTok{scroll_box}\NormalTok{(}\DataTypeTok{width=}\StringTok{'500px'}\NormalTok{,}\DataTypeTok{height=}\StringTok{'400px'}\NormalTok{)}
\end{Highlighting}
\end{Shaded}

All secchi parameter and unit combinations

parameter

units

count

Depth, Secchi disk depth

m

1736036

Depth, Secchi disk depth

ft

386408

Depth, Secchi disk depth

cm

116596

Depth, Secchi disk depth

in

54382

Depth, Secchi disk depth (choice list)

NA

39281

Depth, Secchi disk depth

NA

12037

Secchi Reading Condition (choice list)

NA

864

Secchi Reading Condition (choice list)

None

643

Water transparency, Secchi disc

in

559

Depth, Secchi disk depth (choice list)

428

Depth, Secchi disk depth (choice list)

m

208

Depth, Secchi disk depth

mg

187

Depth, Secchi disk depth

ft/sec

20

Depth, Secchi disk depth (choice list)

None

13

Depth, Secchi disk depth

deg F

6

Depth, Secchi disk depth

deg C

1

Depth, Secchi disk depth

mi

1

Depth, Secchi disk depth (choice list)

ft

1

\hypertarget{secchi-disharmony}{%
\subsubsection{Secchi disharmony}\label{secchi-disharmony}}

Now that we can see all the units we have we can drop non-depth units
and make a lookup table to convert all units to meters.

\begin{Shaded}
\begin{Highlighting}[]
\CommentTok{#Create a lookup table of units and conversion factors that we want to keep}
\NormalTok{secchi.lookup <-}\StringTok{ }\KeywordTok{tibble}\NormalTok{(}\DataTypeTok{units=}\KeywordTok{c}\NormalTok{(}\StringTok{'cm'}\NormalTok{,}\StringTok{'ft'}\NormalTok{,}\StringTok{'in'}\NormalTok{,}\StringTok{'m'}\NormalTok{,}\StringTok{'mi'}\NormalTok{),}
                        \DataTypeTok{conversion =} \KeywordTok{c}\NormalTok{(}\FloatTok{0.01}\NormalTok{,.}\DecValTok{3048}\NormalTok{,}\FloatTok{0.0254}\NormalTok{,}\DecValTok{1}\NormalTok{,}\FloatTok{1609.34}\NormalTok{))}

\CommentTok{# Do an anti_join to these units so that all units that aren't kept can be highlighted and displayed}
\NormalTok{secchi.disharmony <-}\StringTok{ }\NormalTok{secchi }\OperatorTok
\StringTok{  }\KeywordTok{anti_join}\NormalTok{(secchi.lookup,}\DataTypeTok{by=}\StringTok{'units'}\NormalTok{) }\OperatorTok
\StringTok{  }\KeywordTok{group_by}\NormalTok{(units) }\OperatorTok
\StringTok{  }\KeywordTok{summarize}\NormalTok{(}\DataTypeTok{count=}\KeywordTok{n}\NormalTok{())}


\NormalTok{secchi.disharmony }\OperatorTok
\StringTok{  }\KeywordTok{kable}\NormalTok{(.,}\StringTok{'html'}\NormalTok{,}\DataTypeTok{caption=}\StringTok{'The following secchi measurements}
\StringTok{        were dropped because the units do not make sense'}\NormalTok{) }\OperatorTok
\StringTok{  }\KeywordTok{kable_styling}\NormalTok{() }\OperatorTok
\StringTok{  }\KeywordTok{scroll_box}\NormalTok{(}\DataTypeTok{width=}\StringTok{'500px'}\NormalTok{,}\DataTypeTok{height=}\StringTok{'400px'}\NormalTok{)}
\end{Highlighting}
\end{Shaded}

The following secchi measurements were dropped because the units do not
make sense

units

count

428

deg C

1

deg F

6

ft/sec

20

mg

187

None

656

NA

52182

\hypertarget{secchi-harmony-in-meters}{%
\subsubsection{Secchi harmony in
meters}\label{secchi-harmony-in-meters}}

\begin{Shaded}
\begin{Highlighting}[]
\CommentTok{#Join secchi by unit name and then multiply by conversion factor to get meters}
\NormalTok{secchi.harmonized <-}\StringTok{ }\NormalTok{secchi }\OperatorTok
\StringTok{  }\KeywordTok{inner_join}\NormalTok{(secchi.lookup,}\DataTypeTok{by=}\StringTok{'units'}\NormalTok{) }\OperatorTok
\StringTok{  }\KeywordTok{mutate}\NormalTok{(}\DataTypeTok{harmonized_parameter =} \StringTok{'secchi'}\NormalTok{,}
         \DataTypeTok{harmonized_value=}\NormalTok{value}\OperatorTok{*}\NormalTok{conversion,}
         \DataTypeTok{harmonized_unit=}\StringTok{'meters'}\NormalTok{)}
\end{Highlighting}
\end{Shaded}

Next easiest is TSS

\hypertarget{tss}{%
\subsection{TSS}\label{tss}}

This \href{https://water.usgs.gov/osw/pubs/WRIR00-4191.pdf}{paper} is
really useful for exploring this data. In this paper, the USGS directly
compares estimates of Suspended Sediment Concentration (SSC) and Total
Suspended Solids (TSS). The primary difference between these methods, as
laid out in this paper, is that SSC estimates the mass of suspended
solids in a sample volume, by drying out the entire sample without
subsampling the water volume. TSS methods often involve some form of
subsampling of the total water volume. The paper highlights that while
many estimates of TSS and SSC are essentially the same, samples with
high sand content show systematic bias in TSS estimates. For our
purposes, we have no apriori way to distinguish samples with high or low
sand, so we have made the choice to assume that measurements of SSC and
TSS are, over the bulk of samples, the same. We use the term ``TSS''
from here on to describe this data that is both SSC and TSS.

\begin{Shaded}
\begin{Highlighting}[]
\CommentTok{#Read in the raw data from '1_wqdata/out'}
\NormalTok{tss <-}\StringTok{ }\KeywordTok{read_feather}\NormalTok{(}\StringTok{'1_wqdata/out/wqp/all_raw_tss.feather'}\NormalTok{) }\OperatorTok
\StringTok{  }\KeywordTok{wqp.renamer}\NormalTok{()}
\end{Highlighting}
\end{Shaded}

\begin{verbatim}
## [1] "You dropped 88556 samples because the sample medium was not labeled as Water"
\end{verbatim}

\begin{Shaded}
\begin{Highlighting}[]
\CommentTok{#Summarize by characteristic name and unit code}
\NormalTok{tss }\OperatorTok
\StringTok{  }\KeywordTok{group_by}\NormalTok{(parameter,units) }\OperatorTok
\StringTok{  }\KeywordTok{summarize}\NormalTok{(}\DataTypeTok{count=}\KeywordTok{n}\NormalTok{()) }\OperatorTok
\StringTok{  }\KeywordTok{arrange}\NormalTok{(}\KeywordTok{desc}\NormalTok{(count)) }\OperatorTok
\StringTok{  }\KeywordTok{kable}\NormalTok{(.,}\StringTok{'html'}\NormalTok{,}\DataTypeTok{caption=}\StringTok{'All tss parameter and unit combinations'}\NormalTok{) }\OperatorTok
\StringTok{  }\KeywordTok{kable_styling}\NormalTok{() }\OperatorTok
\StringTok{  }\KeywordTok{scroll_box}\NormalTok{(}\DataTypeTok{width=}\StringTok{'500px'}\NormalTok{,}\DataTypeTok{height=}\StringTok{'400px'}\NormalTok{)}
\end{Highlighting}
\end{Shaded}

All tss parameter and unit combinations

parameter

units

count

Total suspended solids

mg/l

2829187

Suspended Sediment Concentration (SSC)

mg/l

1156062

Suspended sediment concentration (SSC)

\%

719493

Total suspended solids

NA

235106

Fixed suspended solids

mg/l

220774

Suspended sediment concentration (SSC)

mg/l

12041

Fixed suspended solids

NA

9354

Suspended Sediment Concentration (SSC)

\%

6758

Suspended Sediment Concentration (SSC)

NA

5393

Total suspended solids

\%

4735

Suspended sediment concentration (SSC)

NA

3489

Total suspended solids

ppm

1680

Total suspended solids

tons/day

529

Total suspended solids

ug/l

476

Total suspended solids

35

Total suspended solids

kg

29

Total suspended solids

None

16

Total suspended solids

count

1

Total suspended solids

NTU

1

\hypertarget{tss-methods}{%
\subsection{TSS methods}\label{tss-methods}}

\hypertarget{tss-analytical-methods}{%
\subsubsection{TSS analytical methods}\label{tss-analytical-methods}}

Unlike with secchi disk depth measurements, there are a variety of
analytical methods available to measure TSS. Many of these should
provide similar results, but some can be used to highlight potential
erroeneous data entry (Phosphorous content should not be a method of TSS
calculation), but many of them are simply labeled by a state or federal
protocol number, making trimming and evaluating these various methods
difficult. Below is a table of all possible analytical methods and their
count

\begin{Shaded}
\begin{Highlighting}[]
\NormalTok{tss }\OperatorTok
\StringTok{  }\KeywordTok{group_by}\NormalTok{(analytical_method) }\OperatorTok
\StringTok{  }\KeywordTok{summarize}\NormalTok{(}\DataTypeTok{count=}\KeywordTok{n}\NormalTok{()) }\OperatorTok
\StringTok{  }\KeywordTok{arrange}\NormalTok{(}\KeywordTok{desc}\NormalTok{(count)) }\OperatorTok
\StringTok{  }\KeywordTok{kable}\NormalTok{(.,}\StringTok{'html'}\NormalTok{,}\DataTypeTok{caption=}\StringTok{'All tss analytical methods and their count'}\NormalTok{) }\OperatorTok
\StringTok{  }\KeywordTok{kable_styling}\NormalTok{() }\OperatorTok
\StringTok{  }\KeywordTok{scroll_box}\NormalTok{(}\DataTypeTok{width=}\StringTok{'600px'}\NormalTok{,}\DataTypeTok{height=}\StringTok{'400px'}\NormalTok{)}
\end{Highlighting}
\end{Shaded}

All tss analytical methods and their count

analytical\_method

count

NA

2253048

Total Suspended Solids in Water

660620

Non-Filterable Residue - TSS

648298

Residue by Evaporation and Gravimetric

249769

Sediment conc by filtration

161882

Sus solids, wat, 105C,wt (NWQL)

108994

analyticTSS\_101

105332

Computation by NWIS algorithm

84571

Sediment conc from size analysis

74063

Sediment conc by evaporation

72438

Volatile Residue

68397

analyticTSS\_GRAVIMETRIC; DRIED AT 103-105 C

67208

Wet sieve

63916

RESIDUE, TOTAL NONFILTRABLE (MG/L)

53434

Total Suspended Solids

46581

160.2

32124

V A tube

28102

Residue

27186

Fixed and Volatile Solids in Water

26565

SM2540D

24720

SM 2540D

22725

Unkn Lab Proc

20850

Dry sieve

20609

EPA 160.4

17934

UNKOWN

15259

UNKNOWN

14844

Effluent monitoring sample

14835

Historic

14695

Historic Procedure

13486

SM 2540 D

11511

Gravimetric Filtration Method; Dried At 90-105

10224

Suspended-Sediment in Water

8892

EPA 160.2

8484

!Generic Placeholder - To Be Determined

8339

Standard Methods for the Examination of Water and Wastewater

8031

Residue, NF,105C (SM2540D/DODEC)

7776

RESIDUE, FIXED NONFILTRABLE (MG/L)

7726

Pipet

6224

Analytical procedure not specified

5729

General Listing of Field and Lab Analytical Procedures for Manatee
County

4943

Total Suspended Solids (TSS)

4885

L01

4847

SSC by filtration (D3977;WI WSC)

4538

Residue Non- filterable (TSS)

4247

Laboratory Procedures for Water Quality Chemical Analysis

4120

Sedigraph

3939

QA Plan \#900456

3044

SUSPENDED SEDIMENT CONCENTRATION

2883

Percent Of Suspended Sediment Particles Passing Through 0. 062 Mm Sieve

2797

TSS, 105C (SM 2540D; WI WSC)

2566

Sediments

2539

Nutrient analyses from water samples

2396

TSS, 105C (EPA 160.2 /DODEC)

2382

Metals in Water by ICP-AES

2368

Lake Trafford

2267

Total Sediment

2163

Nutrient analyses

2157

Total Dissolved Solids in Water

2056

DATABASE CALCULATED FSS - METHOD 1 - MDL

1924

Solids, Suspended and Volatile Suspended

1852

Suspended solids, 105C,grav (PA)

1707

Filterable Residue - TDS

1706

Solids, residue at 105°C

1478

USGS\_HIST

1381

Suspended SedimUSEPAent Concentration (SSC)

1371

Other or Unknown Procedure

1361

EPA160.4

1355

Legacy Guam EPA Analytical Procedures

1310

TOTAL SUSPENDED SOLIDS

1302

ASTM D3977

1270

TOTAL SUSPENDED SOLIDS, DRIED AT 103-105C

1176

See QA Plan for this Project

1165

Fines

1140

Sands

1137

ASTM 3977C, SEDIMENT CONCENTRATION

1125

TSS, dried 103-105C (SM2540D)

935

Gravimetric Evaporation Method; Dried At 90-105 Degrees C

835

Chlorophyll a-b-c Determination

805

L01\textasciitilde{}SSC\_\%FINE

781

Non-filterable residue by filtration and drying

768

R10PORTLANDHARBOR Method

768

LAKE COUNTY QUALITY SYSTEMS MANUAL

763

Residue, total nonfilter, 105C S

754

Total Solids Dried 103-105C in Water

744

L01\textasciitilde{}SSC\_\%SAND

731

total suspended solids

649

TSS

605

Unknown Method or Procedure

574

Residue,total nonfilterable,105C

567

TOTAL NONFILTRATABLE RESIDUE

554

Total suspended solids

541

Nitrogen in Water

521

STANDARD METHODS 2540D - TSS

497

Total Residue

491

Hawaii historic procedures for Legacy STORET

482

Estero Bay Aquatic Preserve tributary sampling

464

USGS I-3765-85

445

Field Office procedures

435

Harrel's TSS Procedure at BITH

391

Settleable Solids in Water

382

Total Phosphorus After Block Digestion

373

SM182540D

371

EPA160.2

369

Southern Ute Tribe Standard Operating Procedures

361

Alkalinity in Water by Titration

349

SOLIDS, SUSP. - RESIDUE ON EVAP. AT 180 C (MG/L)

346

Inorganic Anions by Ion Chromatography

331

Total Suspended Solids Dried at 103-105

326

NFS or Nonfilterable Solids

322

DATABASE CALCULATED FSS - METHOD 1 - 1 MDL

284

DATABASE CALCULATED FSS - METHOD 1 - 1/2 MDL

284

Turbidity by Nephelometry

277

Total Kjeldahl Nitrogen by Colorimetry

269

ASTM D3977M

239

UNK-CH2M

236

Conductance

227

Silica in Water by Colorimetry

215

Susp solids, 105C (160.2;CO WSC)

198

Total Suspended Solids determined by EPA Standard Method 160.2

191

Lower Brule Quality Assurance Project Plan

190

Bunker\_USGS Analytical Methods

184

Total Organic Carbon by Combustion-Infrared Method

183

WRS 14B.1

173

Cheyenne River Sioux Tribe Quality Assurance Procedures

160

Phosphorus by Colorimetry

160

Suspended solids, freeze dried

160

Procedure not Required when Result Entered into National Database

156

Nitrate, Total N, Total P, Si, TSS, Ammonia N, Chlorophyll `a';

154

Sus solids, wat, 110C,wt Ocala)

148

Standard Operation Procedure

137

Legacy STORET migration; analytical procedure not specified

121

Temperature

117

Confederated Salish and Kootenai Tribes Quality Assurance Project Plan

98

pH

98

EPA 160.2M

97

Ammonia in Water Using Automated Phenate Method

92

Residue, fixed nonfilterable

91

Sediment, wet-sieving-filtration

90

Nitrate-Nitrite Nitrogen by Colorimetry

82

ASTM 3977B, TOTAL SUSPENDED SOLIDS

72

SM 2540 B

72

Turbidity in Water

69

Nutrient water sample analyses from Santa Monica Bay

63

Procedure is unknown

63

Sediment size by photo optics

60

Suspended solids after ignition

59

TSS, dried 103-105C (I-3765-85)

59

SM 2540 E

55

Water Quality Measurements by Standard Methods, 14th Edition

47

Nutrient water sample analyses from Morro Bay

46

B W tube

44

5-Day Biochemical Oxygen Demand

42

Dissolved Oxygen Using an ISE

33

Sediment conc by centrifuge

32

Sed conc by acoustic backscatter

27

SLD04

27

SED10

26

HARDNESS IN WATER BY EDTA TITRATION

17

PH TEMPERATURE BY SM 2550B-00

15

Silica in Water by Spectrophotometry- Heteropoly Blue Method

15

TOTAL FILTRATABLE RESIDUE

13

USGS\_UNKN

12

LABORATORY CALCULATION

11

Quality Assurance Project Plan

10

Residue, fixednonfilterable (CO)

10

Dissolved Oxygen by Winkler Technique

9

Sediment size by sedigraph

9

Phosphorus by Two Reagent Colorimetry

8

SED16

8

STANDARD METHOD 4500 FOR PH

6

Ammonia in Water - Flow Injection Analysis

5

SED09

5

Determination of Inorganic Anions by Ion Chromatography

4

Metals in Waters by ICP/MS

4

Persufate Method for Total Nitrogen

4

Sediment by pressure difference

4

Total Nonfilterable Residue Solids

4

Total Organic Carbon in Water- Ultraviolet Oxidation Method

4

FIELD PARAMETERS

3

METALS

3

Standard Analytical Procedure

3

Total, Fixed and Volatile Solids

3

Enzyme substrate assay for measuring total coliforms and E. coli
(ONPG-MUG test or CPRG-MUG test)

2

SM 2540 D v20

2

Susp solids,105C(160.2;CO;HFMAN)

2

TOTAL DISSOLVED SOLID DRIED AT 180 DEGREES CENTIGRAGE

2

5 Day Biochemical Oxygen Demand

1

Ammonia Nitrogen by Colorimetry

1

CONDUCTIVITY LABORATORY METHOD

1

Nephelometric Method

1

Sediment size by wet sieve

1

STANDARD METHOD 2320B

1

STANDARD METHOD 4500-H+

1

Many of these analytical methods are sensible and we can keep the
majority of the data. One glaring issue is that more than 2 million
observations have an analytical method of NA. What should we do with
these samples? Throw out half of our data because the method is not
verifiable? Or keep it knowing that some of the data might be using
incompatable methods.

\hypertarget{tss-sample-methods}{%
\subsubsection{TSS sample methods}\label{tss-sample-methods}}

We can breakdown our TSS data into other categories as well, above we
describe the breakdown of analytical methods, but what about the various
methods that might be used to actually collect the water from the water
body (here called ``sample\_method'')

\begin{Shaded}
\begin{Highlighting}[]
\NormalTok{tss }\OperatorTok
\StringTok{  }\KeywordTok{group_by}\NormalTok{(sample_method) }\OperatorTok
\StringTok{  }\KeywordTok{summarize}\NormalTok{(}\DataTypeTok{count=}\KeywordTok{n}\NormalTok{()) }\OperatorTok\StringTok{ }
\StringTok{  }\KeywordTok{arrange}\NormalTok{(}\KeywordTok{desc}\NormalTok{(count)) }\OperatorTok
\StringTok{  }\KeywordTok{kable}\NormalTok{(.,}\StringTok{'html'}\NormalTok{,}\DataTypeTok{caption=}\StringTok{'All tss sample methods and their count (678 methods)'}\NormalTok{) }\OperatorTok
\StringTok{  }\KeywordTok{kable_styling}\NormalTok{() }\OperatorTok
\StringTok{  }\KeywordTok{scroll_box}\NormalTok{(}\DataTypeTok{width=}\StringTok{'600px'}\NormalTok{,}\DataTypeTok{height=}\StringTok{'400px'}\NormalTok{)}
\end{Highlighting}
\end{Shaded}

All tss sample methods and their count (678 methods)

sample\_method

count

USGS

1702606

Surface

322724

Grab sample. Submerge and fill a water sampling vessel, or sample
directly into the sample bottle provided by the an\ldots{}

208770

Point sample

184816

Equal width increment (ewi)

177421

Water Grab Sampling

146899

Grab

128011

G

93474

GRAB

89945

Ambient Water Quality Monitoring QAPP and SOP

84467

Total Suspended Solids/Volatile Suspended Sample Filtration, Processing
\& Storage T

82854

Unknown

68398

Multiple verticals

61189

GRAB SAMPLE

55010

SJR-SOP

54918

Equal discharge increment (edi)

51762

Total Suspended Solids/Volatile Suspended Sample Filtration, Processing
\& Storage

49191

Grab Sample

46873

ADEM SOPs-2000 Series-Surface Water

44394

STD\_SC

43858

Water\_Grab

41246

GRAB

41147

Grab sample (dip)

38035

SOP For The Collection And Preservation Of Stream And River Grab Samples
For Chemical And Biological Analysis.

37311

STANDARD

36280

SuspSed; Pumping - stream sample using a pumping mechanism

35832

Chesapeake Bay Program

35522

Water Grab Sample

33391

OEPA Surface Water Sample Collection Method

33110

Standard UHL Sampling Procedure - Grab Samples

29319

Total Suspended Solids/Volatile Suspended Sample Filtration, Processing
\& Storage B

29030

SOP

27974

WATER GRAB SAMPLE

27739

Grab water sample taken from a river by using a bottle

27471

Virginia CBP Non-Tidal Tributary Monitoring Program

26492

Grab Sample Collected With A Stainless Bucket

26407

Single vertical

26028

HISTORIC

23895

Automated Water Sampler

23186

WQM Sample collection

22671

Water quality grab sampling.

21588

FL DEP Field Sample Collection DEP SOP 01-001

20919

WATER SAMPLE

20548

UHL001

20235

Grab Sampling

19765

Lake depth point sampling. Lake water is sampled at a discrete depth in
the water column using a vertical Kemmerer- \ldots{}

17776

Lake surface 2m depth-integrated sampling. Lower a 2-meter-long,
2-inch-diameter PVC pipe vertically into the water, \ldots{}

17711

Generic SJRWMD Sample Collection Procedure

17358

NA

17028

ISU Lake Sample Collection Procedure

16081

SWFWMD SOP's for the Collection of Water Quality Samples

15774

FACILITY EFFLUENT SAMPLE

14835

R

14365

SuspSed;Partial Depth,depth integrated,part of single vert.

14070

Grab Water Sample

14004

SuspSed; Box-single ver, depth-int, attached to structure

13729

Unknown, Historic Data, Migrated from STOREASE

13486

Composite - Multiple point samples

13287

Standard Method

11108

COLLECT01

10460

Grab sample

10369

Standard Routine Sample Collection

10351

SP-001

10267

FDEP SOP Surface Water Sample

10211

NPS\_LEGACY

9618

Multiple verticals, non-isokinetic, equal widths and transit rate

8517

NOT PROVIDED

8339

UNKNOWN

8290

SOP-1

8269

Collier County Water Sampling Collection Procedure

8182

SM 1060B

7977

SWAMP Sample Collection Procedure

7803

Land Collections Directly from Water Source

7793

Sample-Routine

7371

Total Suspended Solids/Volatile Suspended Sample Filtration, Processing
\& Storage A

7304

Photic

6611

Composite sample, flow-weighted/flow-paced with auto-sampler

6604

Unspecified grab

6520

SOP

6467

AZDEQ Standard Operating Procedures

6294

LEGACY SCP

5954

Surface Water Sample Collection QAPP

5932

Water Sampling, grab

5782

EPA METHOD

5723

kemmerer bottle

5674

IL\_EPA

5606

CREATEW-2

5500

Grab-Direct to Sample Container

5362

Land Station Collection from Bridge

5352

COMPOSITE

5184

Grab Sample Collected With A Water Bottle

5138

DH59 composite

5104

Weighted bottle

5083

UHLLAKE

5074

Water quality grab sampling

4935

Surface Water QAPP

4839

Grab water sample taken from a river using a bottle

4725

FDL QAPP

4708

STREAM\_1

4674

WPMP Stream Sampling SOP

4533

RAMP Sample Collection Procedure

4363

Bucket/Bottle Grab

4247

FPSWP

4170

Total Suspended Solids/Volatile Suspended Sample Filtration, Processing
\& Storage K

4161

water sample collection

3980

NCA Water Sample

3937

Peristaltic pump

3930

UHL Lake sampling procedure

3922

Other

3827

GRAB-1

3814

RIGRABOT

3774

Standard Grab Sampler

3620

STD\_LM

3617

Grab Sample Collection

3544

Van Dorn bottle

3477

Boat Station Collections

3426

Grab water sample taken from a reservoir by using a Van Dorn bottle

3333

SOP-3

3282

GRAB01

3221

water bottle

3203

ASIS\_GRAB

3182

Grab water sample taken from a lake by using a Van Dorn bottle

3085

Coastal 2000 Program

2947

Chemistry - Sample

2867

RIBS-WCOL

2753

SURFACE WATER GRAB SAMPLE

2730

FPRMP

2692

DEP-SOP-001/01 FS2100 Surface Water Sampling

2679

DNRLAKE

2676

NCA Standard WQ Sample

2660

Shenandoah Fish Kill Task Force Study

2642

DEP-SOP-001/01 FS2100

2629

AMBIENT AND VRAP SAMPLING PROCEDURES

2555

Niskin water sampler

2505

Routine Water Quality Samples

2410

Composite Effluent Sample

2407

USGS Collection Method

2373

Go-Flo sampling bottle

2157

Water samples taken from a reservoir by using a Van Dorn bottle and
composited

2116

Suction lift peristaltic pump

2059

R10BUNKER

2033

SRBC Standard Grab Sample Method

1986

Kemmerer composite

1970

LAGRBVAN

1965

THE COLLECTION AND PRESERVATION OF STREAM AND RIVER GRAB SAMPLES FOR
CHEMICAL AND BIOLOGICAL ANALYSIS.

1931

REGRAVAN

1898

Samples Collection Procedures QAPP

1833

Blank water sample for QAQC purposes using a bottle

1783

Intermediate

1781

Timed sampling interval

1762

grab sampling

1755

USEPA

1733

EWI-CHURN

1732

GEPAMP022

1688

Grab-Using Sampler

1688

GRYN\_DH81

1665

Iowa Geological Survey - general procedures for grab samples

1658

Standard Operation Procedure

1652

Vacuum Filtration

1604

Brass Kemmerer sample

1563

Grab Sample-Grabber Bottle

1555

STD\_SP

1505

DRBC SRMP QAPP or SOP

1503

Delaware River Basin Commission QAPPs or SOPs

1486

Intermediate Grab

1486

IRONMT

1450

Water samples taken from a lake by a Van Dorn bottle and composited

1445

Sample-collection method not available

1394

EWI

1368

GRAB-01

1358

Marine Offshore Water Column Sample

1331

Guam EPA Legacy Sampling Procedures

1310

RECOMVAN

1287

Van Dorn sampler

1284

Replicate grab water sample taken from a river by using a bottle

1240

01

1223

Subsurface grabs equidistant across stream

1173

BB-002

1171

Potomac River Embayment Study

1161

STANDARD GRAB

1160

Standard sampling method

1144

Not Applicable (N/A)

1141

Fort Peck Tribes Sample Collection Procedure

1140

INVALID DATA SET QUALITY ASSURANCE FAILURE

1136

WATER

1132

South, SF Shenandoah and Shenandoah River Mercury Study

1131

Water sample taken from a river by using an integrated suspended
sediment sampler

1120

CSKT\_QAPP

1111

SCP-001

1093

ACT

1051

WATER1

1039

GP

1016

Large Rivers Water Quality Monitoring Protocol

1006

Ute Mountain Ute Water Quality Standard Operating Procedures

980

G

960

Surface sample

943

LACOMVAN

933

Fort Belknap - Standard Sample Collection Procedure

914

Discrete grab with bottle

913

Water Quality Sampling

909

Collecting Water Quality Samples

897

R10MIDNITE

894

Integrated Vertical Profile

891

Project Sampling Plan

881

Water Sampler Standard Operation Procedure

877

WQ-1

859

The collection and preservation of stream and river grab samples for
chemical and biological analysis.

858

Integrated Grab Water Sample

847

COLLECT1

845

ROUTINE

836

SP-01

834

Depth Integrated Water Sample

832

Texas Commission on Environmental Quality, Surface Water

808

FDL\_SCP

805

FCPC\_WQP

789

Spring Branch TMDL (VAP-K32R)

785

Grab sample collection

768

Portland Harbor Sampling Methods

768

Grab water sample taken from a river using a bucket

747

YLR001

745

TAT\_SCP

738

Chemical and Physical Monitoring of Water Resources

736

Van Dorn Water Sample

734

grab sample

725

ROUTINE

704

NM Surface Water Quality Bureau QAPP

657

UHL-Composite Sampling Procedure for TMDL

653

Lake surface depth-integrated sampling other than 0-2m

652

CDI

648

Equal Width Increment (EWI) Sample Collection

640

Elizabeth, Tidal James, New Rivers, and Mountain Run PCB TMDL Study

636

EPA R4 SESD FBQSTP

633

GB

626

CPCRI TRACE ELEMENTS MAJOR ANIONS CATIONS MAINSTEM CLINCH RIVER

625

General

620

Grab-Unknown Method

615

PORE\_WQSAM

615

General Water Quality Procedures

611

GRAB001

607

See Project QA Plan

600

Bucket

574

Stainless Steel Bucket

573

DRBC Project QAPP or SOP

571

Composite water sample taken from a river with an ISCO automated sampler

570

AUTOSAMPLE

568

Little Traverse Bay Bands of Odawa Indians Quality Assurance Project
Plan

562

Oglala Sioux Tribe Sample Collection Procedure

555

SW1-WS

549

Eagle Mine Compliance Monitoring

532

Follow Collier County Pollution Control Standards

529

WPMP Lake Sampling SOP

505

North River Tributary TMDL Implementation

500

Prairie Island QAPP

499

2010 PCB Fish Consumption Study

496

Elizabeth and Upper James Tidal PCB TMDL Study

492

Surface Water Sampling

489

Bedload, single equal width increment (SEWI)

482

Historic Hawaii Sample Collection methods for legacy STORET

482

Southern Ute Tribe Sample Collection Procedures

481

Lake Sampling SOP (6/1/2010) - Depth Sample

475

SOP For The Collection And Preservation Of Lake Or Non-Wadable Wetland
Water Column Samples For Chemical Analysis.

463

Field Surface Water Collection

456

303D-WAT

451

UT to Chickahominy R TMDL

449

Grab Sample (Laboratory Data)

445

Unspecified

445

NPS\_DIS

435

RIINTINT

434

Chippewa Cree 106 QAPP Sample Collection Procedure

431

RICOMAUT

430

Water: 2 L composite for onsite analyses

424

Composite Influent Sample

420

Quality Assurance mechanism for the Stockbridge-Munsee Community

418

See QA Plan for this Project

417

BB-005

415

Lac du Flambeau Tribe

396

QAPP

392

Harrel's WQ Sample Collection Procedures

391

Lake Ecology

381

Equipment Blank

379

Dan River Fly Ash Spill

372

Field Blank

372

NPS\_GRAB

360

Otoe Missouria Tribe

350

DRC

346

GTMGRAB

344

R10BUNEKR

343

Upper Sioux Community Quality Assurance Project Plan

336

Grab Sample Method

330

Surface Water Sampling Method

330

Rivanna River TMDL IP

324

SP-QAPP

323

Cheyenne River Sioux Tribe - Standard Sample Collection Proc

322

COLLECT-01

322

Field and Laboratory Methods for Sediment

321

Stream Water Sample

317

SMFY04QAPP

313

Water samples taken from a lake by a water bottle and composited

311

UAA Quality Assurance Project Plan

306

WHITBRST

303

Whitebreast Snapshot sampling procedure

303

Composite Sample Collected With A DH-81 Depth-Integrating Vertical
Sampler

296

Water grab sample taken from a storm sewer by using a bottle

289

Lower Brule Tribe Sample Collection Procedure

285

GB-SCP

283

QAPP Sample Collection Procedure

278

Jackson River Benthic TMDL

276

1

270

MCCBCED

270

Velocity integrated

267

CCERIV

263

ISU CCE Standard Procudures for Sampling Rivers

263

Water Collection

263

Point Surface Water Sample

261

Laurel, Fridley Run, Little Stony Creek Benthic TMDL IP

260

PRO James River Bacterial TMDL Study

260

LAINTVAN

256

GLKNRVWQ

249

Unknown Sample Collection Procedure

249

Cooks Creek Blacks Run TMDL IP Monitoring

243

POTW HARDNESS AND THE EFFECT ON TOTAL AND DISSOLVED METALS

242

Composite Sampling

239

SCC-SOP

239

Laboratory Samples

235

Standard South Dakora equipment

231

Thief sample

231

COLLECT02

230

Meadow Creek and Shencks Branch Benthic TMDL

220

WQCD Stream Sampling Procedure

217

Procedure not Required when Result was Entered into National Database

216

Bottom

214

LAKE\_1

213

WATER GRAB SAMPLE 2005 Guidelines and Procedures

213

Grab water sample taken from a canal by using a bottle

212

SCITQAPP2013-2017

211

Water Sampler

211

HC041206

207

Composite sample, time-paced auto-sampler. Automatic composite sampling
at regular time intervals.

205

Garden Creek Q04

204

Composite sample, flow-weighted/time-paced with auto-sampler

198

Water samples taken from vertical profile of a lake by using a Van Dorn
bottle and composited

197

Discharge integrated, centroid

191

HANALEI

191

Grab water sample taken from a lake by using a bottle

189

POINT

185

South Fork and North Fork Catoctin Creek TMDL monitoring (A02)

184

GP

182

Water Quality Grab Sample Method

177

Sampling Collection

176

GRAB-001

175

Isco Flow Weighted Composite

171

Broad Run Benthic TMDL (VAN-A09)

168

MM-PMP-SOP

166

IASNAPSHOT

165

IOWATER Snapshot Sampling Procedure

165

Sandia Field Collection Measurements

165

Discrete sample, time-paced with auto-sampler

161

PFIESTERIA PROGRAM

161

UIT Quality Assurance Plan

160

HIDOH Sample Collection

154

SFAN\_SOP\_8

154

Auto Sampler

153

WRPT QAPP

153

Standing Rock Standard Collection Procedure

152

Clinch Powell Clean Rivers Initiative

150

Sokaogon Chippewa Community Quality Assurance Project Plan

150

Roanoke River Watershed TMDL

148

Pigg River Watershed TMDL

147

Watershed Projects with Lake and Stream Sites

147

Standard Collection Procedure

145

MM-PDDN-SOP

144

Composite sample, flow-triggered, time-paced, auto-sampler. Automatic
sampling at regular time intervals triggered \ldots{}

141

Long Meadow and Turley Creek Benthic TMDL

138

Opequon Creek Abrams Creek TMDL IP Monitoring

135

Jeffries Branch Benthic TMDL (VAN-A04)

134

NC\_SCP

134

Stony Run and Deep Run Henrico Benthic TMDL

134

Sampling and Analysis Plan/ Quality Assurance Project Plan

132

Cunningham Creek TMDL

131

Powell River (VAS-P17R) Benthic and Bacteria TMDL

131

SAMPLE COLLECTION METHOD

131

Difficult Run Watershed for Benthic TMDL

130

New River Valley TMDL (Crab, Back, Peak)

128

ACF

127

LTBBQAPP

127

South River Intensive Water Column Mercury Sweep

126

Depth integrated sample with DH81 isokinetic sampler

123

Grab sample at water-supply tap

123

Integrated Water Sample

122

Lake Depth Integrated Water Sample

119

Shakopee Mdewkanton Sioux Community QAPP

119

Composite sample with auto-sampler

118

Clinch River:Dominion-DEQ

117

Water: 2 L composite for laboratory analyses

117

Water bottle

114

SMSCQAPP

113

water grab sampling

113

Grab water sample taken from a reservoir by using a bottle

108

UHL002

108

Composite water sample taken from a storm sewer with an ISCO automated
sampler

104

Little Calfpasture River TMDL

104

Buffalo River E. coli TMDL Study (H11R, H12R)

102

FB

102

Special Study Sand Branch (Loudoun Co)

102

AQUA

101

Accotink Creek Watershed for Benthic TMDL

100

H\&C QAPP

100

Stream Grab Sample

100

Grab sample at Tap(s) on a Dam

99

Suction pump

98

Critical basket from bridge

96

Beaver Creek TMDL

92

ON\_QAPP

91

SHIL\_PCA\_2

90

2008 TMDL IP Monitoring

89

Microbiology Analysis, Critical basket from bridge

89

SCITQAPP2009

88

Blackwater River Franklin County TMDL

87

Clean Hands Dirty Hands

87

Columbus Water Works Sample Collection SOP

87

GOGA\_GRAB

87

STP Autosampler

86

West Strait Creek TMDL

86

SAP QAPP FOR GW AND SW MONITORING AT THE WEST EXPANSION OF THE PAGE
REPOSITORY

84

CAGRABOT

82

Grab sample collected in response to an event.

82

Red Cliff Water Quality Monitoring

82

Summerduck Run Benthic TMDL (VAN-E10)

82

KBIC QAPP

81

106 QAPP

80

Bacteria Sampling

80

Composite sample (other)

80

Composite-Vertical-Discrete-Sampler

80

SuspSed;Single-stage,nozzle at fixed stage,passively fillng

80

MO Department of Natural Resource Water Sampling Protocol

79

South Anna Benthic TMDL (VAN-F01)

76

Tripps Run and Holmes Run TMDL monitoring (A13)

76

Crow Nation Sample Collection Procedure

75

TMDL sampling QAPP

75

Composite sample from multiple locations on a waterbody, combined with a
churn splitter.

74

GP QAPP

74

Surface water collection

74

Surface Water Sampling SOP

74

Continuous-flow sampling, Clean Hands technique. Lower teflon collection
tube to a representative depth of the water\ldots{}

73

Grab water sample taken from the bottle of the ISCO automated sampler
from a river

73

Holmans Creek TMDL Implementation

72

Mountain Run Benthic TMDL (VAN-E09)

72

SCRO PCB Roanoke River 2007

72

SHIL\_PCA\_3

71

Rockfish Bacteria and Benthic TMDL IP H15R and H16R Nelson County

70

SOP For The Collection Of Lake Or Non-Wadable Wetland Water Samples
Using 6-Foot Depth Integrated Column Sampler.

70

Smith Creek, Mountain Run, Fridley Run TMDLs

69

POLSWATER\_WQX\textasciitilde{}Field Probe

67

FLOYDCED

66

Laurel Fork (VAS-N37) TMDL

66

Little Calfpasture Turbidity Study

66

STORM SAMPLER

66

CCELAK

63

ISU CCE Standard Procedures for Sampling Lakes

63

Sample Method Unknown

63

Stroubles Creek TMDL

63

WPMP Estuary Sampling SOPng

61

Bluestone River and Tributary PCB and Chlordane TMDL Study

60

Bluestone RiverN36/N37 TMDL

60

Clinch and Tribs. (P03, P04)

60

Spout Run Bacteria TMDL

60

Stream Condition Index and Fixed Trend Monitoring Protocols

59

North Creek Benthic TMDL Stressor Study

58

North Fk. Holston and Tribs.

58

Rivanna NFRivanna Benthic TMDL

58

Sample Collection Method

57

Buffalo River Benthic TMDL Study

56

2013125

52

Pound River, North \& South Forks Q13R

52

Straight Creek (VAS-P20R) Benthic and Bacteria TMDL

51

Wadeable Stream Nutrient Criteria pilot Project

51

WICR\_UMCW

51

NEWSQAPP

50

SD WRAP

50

Special Study Quantico Creek 2015 \& 2016

50

Standard Sampling Methods

50

Composite

49

Lake Grab Sample

48

Upper Basin Field Studies Phase 1 Investigation

48

Water Sampling with Collapsible Container by Dasher

47

Meter

46

MORRTRAMA

46

SHIL\_PCA\_1

45

FMYN

44

North Fork Powell (VAS-P20) Benthic and Bacteria TMDL

44

OK Corp Comm QAPP

44

Wiyot Tribe Quality Assurance Project Plan

44

North Fork Holston River TMDL

43

SCITQAPP2012

43

QC Field Duplicate-Grab-Unknown Method

42

Standard Grab

42

Grab water sample taken from a facility by using a bottle

41

Match-E-Be-Nash-She-Wish Band of Pottawatomi Indians TWG Quality
Assurance Project Plan

41

2004/2005 VRO BST Studies

40

Hogue Creek TMDL

40

Mill Creek (B48) TMDL

40

PN\_QAPP

40

FWS-2002

37

Hardness,Carbonate

37

SSGRABOT

37

DH-48 Water Sample

36

ON\_SCP

36

Three Creeks TMDL

36

Garden hose composite

35

Lick Creek P10

35

LAINTWAT

34

Maury River benthic TMDL

34

Terry's Run

34

Guest River TMDL

33

SCPN\_L1

33

Devil Fork and Bark Camp Branch Stressor Analysis Study (P12)

32

ISCO Water Sample

32

Roses Creek TMDL

32

SHIL\_PCA\_4

32

DRAFT SAP QAPP FOR GW AND SW MONITORING AT THE PAGE AREA

30

GRAB-CLEAN

30

Tinker Creek Watershed TMDL

30

Water samples taken from a lake by using an Integrated Verticle Tube and
composited

30

Grab Pump

29

Routine Water Quality Sampling

29

Composite-Multiple point samples

28

FISH CONSUMPTION IMPAIRMENT IN THE ROANOKE (STAUNTON) RIVER

28

MM-SMP-SOP

28

QAPP for Cent Gt Plains HW Assessment

28

Ash Camp Creek

27

FLBS Integrated Vertical Water Sample

27

Toms Brook TMDL

26

South River/South Fork Shenandoah Bacteria TMDL

25

Bull Creek and tributaries Q08

24

SHIL\_TSS

24

Water samples taken from a reservoir by using a water bottle and
composited

24

Depositional sample taken to analyze atmospheric phosphorus

23

Lewis Creek TMDL

22

Shallow Water Continuous Monitoring in Tidal Potomac

22

Depth Integrated Ambient procedure

21

Flow Measurement SOP

21

QC Trip Blank - Historical

21

Water sample taken from a storm sewer by using an integrated suspended
sediment sampler

21

LRB\_NUT

20

Rush River

20

Stormwater Sampler and Mounting Kit

20

Middle Creek TMDL P03

19

SAMPLING PROCEDURES FOR VOLUNTEER LAKE ASSESSMENT PROGRAM

19

South Mayo River TMDL

19

Water Grab Sample (2005) Guidelines and Procedures

19

Multi Probe Sonde

18

Red Bank Creek

18

REDW\_DWR05

18

Critical manual grab from bank

17

MSLM

17

Bedload, multiple equal width increment (MEWI)

16

Bull Run \& South Run, and Popes Head Creek Biological Stressors Study

16

Composite w/o Parents

16

HEHO1

16

INDIAN CREEK TAZEWELL COUNTY WATERBODY VAS-P02R

16

Massaponax Creek

16

Naked Creek (Page) Benthic TMDL

16

Niskin Bottle Water Sample

16

Upper James and Lewis Creek PCB Study

16

Appomattox River TMDL

15

FAGRABOT

14

Highland Park Grab Sample

14

VA DEQ Biological Monitoring Program QAPP 2008 final

14

SP-Water

13

Big Reed Island Creek and Tribs TMDL Study (N13, N14, N15)

12

Chestnut Creek (VAS-N06R) Benthic and Bacteria TMDL

12

Chickahominy Mercury TMDL

12

Composite water sample taken from a canal with an ISCO automated sampler

12

Flat, Nibbs, Deep and West Creeks post-IP monitoring plan

12

Goldmine, Beaver, Pamunkey and Plentiful Creeks, Mountain \& Terrys Run

12

Hunting Camp Creek (VAS-N31R) TMDL

12

Little Buffalo Creek special study

12

Looney Creek TMDL

12

Quality Control Sample-Inter-lab Split

12

Robinson River and Little Dark Run Bacteria Study

12

Russell Fork and Tribs TMDL Study (Q09 \& Q10)

12

SJRWMD SAMPLE PROCEDURES FOR VOLUNTEERS

12

Split grab water sample taken from a river using a bottle

12

Standard

12

Standard Grab Method

12

Stock Creek (VAS-P13R) Benthic TMDL

12

Water nutrient samples

12

Water sample taken from a canal by using an integrated flow proportioned
sampler

12

North Fork Holston Chloride Study

11

Standard USVI sampling method

11

Blackwater River PRO Mercury TMDL

10

Clinch River TMDL

10

Quail Run TMDL

10

Quality control blank. No sample collected.

10

Syringe sample

10

Unspecified Standard Grab Sample Procedure

10

Water Bottle

10

Water samples taken from a reservoir by using an Integrated Verticle
Tube and composited

10

Water grab sample taken from a canal by using a bottle

9

BVR SWQAPP

8

PRO PCB Biosolid Study

8

GB

7

James River Mercury Study (H38-H39)

7

LCRQAPP

7

Microbiology Analysis, Critical sampling pole from bridge

7

PWD

7

Sewage sampler

7

Time series composite water sample taken from a storm sewer with an ISCO
automated sampler

7

200

6

Goldmine Creek TMDL for DO (F07)

6

Hardware River Bacteria TMDL

6

Meherrin River watershed bacteria, nutrient, mercury TMDL

6

Monroe Creek Natural Assessment for Low pH and DO

6

Nottoway River watershed bacteria, nutrient, mercury TMDL

6

Reed Creek TMDL

6

Smith Creek TMDL IP Monitoring Study

6

USGS\_THST

6

Composite of grab samples, time-paced. Individual grab samples taken
over time are composited for a single result.

5

DRBC QAPP

5

Grab water sample taken from a reservoir by using a water bottle

5

GROUND WATER GRAB SAMPLE

5

IDNR Fisheries Lake Sampling Technique

5

Lab QC

5

NPS\_DI\_SS

5

Phytoplankton Sample

5

surface- horizontally int

5

UNKNOWN

5

WWMD\_VA - YSI6600V2SondeGauge

5

YSI 6600 V2 Sonde Gauge

5

Critical sampling pole from bridge

4

EOR sample collection procedures

4

LC TRIBAL-QAPP

4

PHOTIC

4

Potomac River PCB TMDL Study

4

Upper New River pH Study

4

Equal Width Increment - Equal Transit Rate

3

Grab sample (dip)

3

Integrated Sampler

3

ISCO Composite Sample - Post-peak

3

ISCO Composite Sample - Pre-peak

3

Jeremy, Gooney, Flint Run and Passage Creek TMDL

3

Microbiology Analysis, Critical manual grab wade

3

No Sample Collection Procedure - No Sample Collected

3

Normal collection of surface water

3

Pamunkey Mercury TMDL

3

Spigot

3

Bedload, unequal width increment (UWI)

2

Benthic invertebrate-net

2

Critical manual grab wade

2

Discharge integrated, equal transit rate (etr)

2

Inland Lakes Water Quality Monitoring Protocol

2

Open-top bailer

2

QAPP

2

USGS\_LIBO

2

VICK\_ACEW

2

Crow Creek Water Collection Procedure

1

Direct Grab

1

Field Blank Sampling Method

1

Flowing well

1

FP

1

FR

1

FS 2100

1

GLKNLKWQ

1

Grab Sample for chemical parameters

1

GRAB-02

1

GRAB-CD

1

Grab-Using Sampler(QC)

1

Macroinvertebrate JAB Sample

1

Periphyton Sampling Gear

1

QC Rinsate Blank

1

SAMPLE PROCEDURES FOR CONDUCTING LAKES ASSESSMENTS

1

Sediment Sieved Sample

1

Soil Sampling

1

Split grab water sample taken from a reservoir by using a Van Dorn
bottle

1

Tribal QAPP

1

UNK

1

Water Grab Sample: Bucket Sampler

1

Water Quality Metal Sampling

1

\hypertarget{tss-depth-of-sampling}{%
\subsubsection{TSS depth of sampling}\label{tss-depth-of-sampling}}

For TSS some sites also have the water depth of sample, which is very
useful for validating whether or not the sample will reflect satellite
observation of the same water parcel. However, most of the data doesn't
have this depth of sampling data and it requires a bit of its own
munging, since the sampling depth comes down in a range of units.

\begin{Shaded}
\begin{Highlighting}[]
\CommentTok{#Define a depth lookup table to convert all depth data to meters. }
\NormalTok{depth.lookup <-}\StringTok{ }\KeywordTok{tibble}\NormalTok{(}\DataTypeTok{sample_depth_unit=}\KeywordTok{c}\NormalTok{(}\StringTok{'cm'}\NormalTok{,}\StringTok{'feet'}\NormalTok{,}\StringTok{'ft'}\NormalTok{,}\StringTok{'in'}\NormalTok{,}\StringTok{'m'}\NormalTok{,}\StringTok{'meters'}\NormalTok{,}\StringTok{'None'}\NormalTok{),}
                       \DataTypeTok{depth_conversion=}\KeywordTok{c}\NormalTok{(}\DecValTok{1}\OperatorTok{/}\DecValTok{100}\NormalTok{,.}\DecValTok{3048}\NormalTok{,.}\DecValTok{3048}\NormalTok{,}\FloatTok{0.0254}\NormalTok{,}\DecValTok{1}\NormalTok{,}\DecValTok{1}\NormalTok{,}\OtherTok{NA}\NormalTok{)) }

\CommentTok{#Join depth lookup table to tss data}
\NormalTok{tss.depth <-}\StringTok{ }\KeywordTok{inner_join}\NormalTok{(tss,depth.lookup,}\DataTypeTok{by=}\KeywordTok{c}\NormalTok{(}\StringTok{'sample_depth_unit'}\NormalTok{)) }\OperatorTok
\StringTok{  }\CommentTok{#Some depth measurements have negative values (assume that is just preference)}
\StringTok{  }\CommentTok{#I also added .01 meters because many samlples have depth of zero assuming they were}
\StringTok{  }\CommentTok{# taken directly at the surface}
\StringTok{  }\KeywordTok{mutate}\NormalTok{(}\DataTypeTok{harmonized_depth=}\KeywordTok{abs}\NormalTok{(sample_depth}\OperatorTok{*}\NormalTok{depth_conversion)}\OperatorTok{+}\NormalTok{.}\DecValTok{01}\NormalTok{)}

\CommentTok{# We lose lots of data by keeping only data with depth measurements}
\KeywordTok{print}\NormalTok{(}\KeywordTok{paste}\NormalTok{(}\StringTok{'If we only kept samples that had depth information we would lose'}\NormalTok{,}\KeywordTok{round}\NormalTok{((}\KeywordTok{nrow}\NormalTok{(tss)}\OperatorTok{-}\KeywordTok{nrow}\NormalTok{(tss.depth))}\OperatorTok{/}\KeywordTok{nrow}\NormalTok{(tss)}\OperatorTok{*}\DecValTok{100}\NormalTok{,}\DecValTok{1}\NormalTok{),}\StringTok{'% of samples'}\NormalTok{))}
\end{Highlighting}
\end{Shaded}

\begin{verbatim}
## [1] "If we only kept samples that had depth information we would lose 75.5 % of samples"
\end{verbatim}

\begin{Shaded}
\begin{Highlighting}[]
\KeywordTok{ggplot}\NormalTok{(tss.depth,}\KeywordTok{aes}\NormalTok{(}\DataTypeTok{x=}\NormalTok{harmonized_depth)) }\OperatorTok{+}\StringTok{ }
\StringTok{  }\KeywordTok{geom_histogram}\NormalTok{(}\DataTypeTok{bins=}\DecValTok{100}\NormalTok{) }\OperatorTok{+}\StringTok{ }
\StringTok{  }\KeywordTok{scale_x_log10}\NormalTok{(}\DataTypeTok{limits=}\KeywordTok{c}\NormalTok{(}\FloatTok{0.01}\NormalTok{,}\DecValTok{10}\OperatorTok{^}\DecValTok{3}\NormalTok{),}\DataTypeTok{breaks=}\KeywordTok{c}\NormalTok{(.}\DecValTok{1}\NormalTok{,}\DecValTok{1}\NormalTok{,}\DecValTok{10}\NormalTok{,}\DecValTok{100}\NormalTok{)) }
\end{Highlighting}
\end{Shaded}

\begin{verbatim}
## Warning: Removed 8592 rows containing non-finite values (stat_bin).
\end{verbatim}

\begin{verbatim}
## Warning: Removed 1 rows containing missing values (geom_bar).
\end{verbatim}

\includegraphics{wqp_data_harmonize_explorer_files/figure-latex/depth breakdown-1.pdf}

If we ignore all these additional data streams and simply assume SSC and
TSS are generally near surface water samples collected with compatible
field sampling and analytical methods. Then we can simply get rid of
samples that have nonsensical units.

\hypertarget{tss-disharmony}{%
\subsubsection{TSS disharmony}\label{tss-disharmony}}

As with secchi disk depth, we expect certain units to be associated with
total suspended solids or suspended sediment concentration. These
include mass per volume measurements like: mg/l, g/l, ug/l and others.

TSS does come with one less obvious parameter which is \%. Any sample
with a \% unit is most commonly a sample where suspended sediments were
split into particle size fractions. The relative proportion of clay,
silt, and sand can have important impacts on the reflectance properties
of water, so this is a useful parameter to keep, though it will require
some exploration, using the additional data column that we relabeld as
``particle\_size.''

\hypertarget{tss-particle-size-fractionation}{%
\paragraph{TSS particle size
fractionation}\label{tss-particle-size-fractionation}}

The table below shows all of the various particle fraction categories
held within the TSS category. About half of the total observations
(760,000) that use ``\%'' as a unit are actually estimating the fraction
of particles that are smaller than sand (\textless{}0.0625). The rest of
the particle fractionation size classes are spread across \emph{29}
other particle fractions. This leaves us with a difficult choice. If we
kept all of this data, we would widen our final dataset by 29 rows, with
very few likely overpasses in a dataset of less than 80k observations
per fraction category before checking for sites that are Landsat visible
and were collected on relatively cloud free days. If we throw away all
of the \% data, we use valuable information that may help explain
variability between sites with similar TSS but different reflectance
values based on the particle size fractionation. Here, we will opt for
an intermediate approach and keep only the \textgreater{} 300,000
observations that simply describe the fraction of sand in a sample
(\textless{}0.0625 mm).

\begin{Shaded}
\begin{Highlighting}[]
\CommentTok{#Select only units for %}
\NormalTok{tss.p <-}\StringTok{ }\NormalTok{tss }\OperatorTok
\StringTok{  }\KeywordTok{filter}\NormalTok{(units }\OperatorTok{==}\StringTok{ '%'}\NormalTok{) }

\CommentTok{#look at the breakdown of particle sizes}
\NormalTok{tss.p }\OperatorTok
\StringTok{  }\KeywordTok{group_by}\NormalTok{(particle_size) }\OperatorTok
\StringTok{  }\KeywordTok{summarize}\NormalTok{(}\DataTypeTok{count=}\KeywordTok{n}\NormalTok{()) }\OperatorTok
\StringTok{  }\NormalTok{knitr}\OperatorTok{::}\KeywordTok{kable}\NormalTok{()}
\end{Highlighting}
\end{Shaded}

\begin{longtable}[]{@{}lr@{}}
\toprule
particle\_size & count\tabularnewline
\midrule
\endhead
\textless{} 0.001 mm & 655\tabularnewline
\textless{} 0.002 mm & 32594\tabularnewline
\textless{} 0.004 mm & 44399\tabularnewline
\textless{} 0.008 mm & 24521\tabularnewline
\textless{} 0.016 mm & 42984\tabularnewline
\textless{} 0.031 mm & 23248\tabularnewline
\textless{} 0.0625 mm & 327984\tabularnewline
\textless{} 0.062mm & 172\tabularnewline
\textless{} 0.063 mm & 15\tabularnewline
\textless{} 0.09 mm & 86\tabularnewline
\textless{} 0.125 mm & 77011\tabularnewline
\textless{} 0.18 mm & 86\tabularnewline
\textless{} 0.25 mm & 68509\tabularnewline
\textless{} 0.355 mm & 80\tabularnewline
\textless{} 0.5 mm & 52289\tabularnewline
\textless{} 0.71 mm & 19\tabularnewline
\textless{} 1 mm & 20561\tabularnewline
\textless{} 1.4 mm & 1\tabularnewline
\textless{} 128 mm & 15\tabularnewline
\textless{} 16 mm & 16\tabularnewline
\textless{} 2 mm & 4189\tabularnewline
\textless{} 256 mm & 15\tabularnewline
\textless{} 3.35 mm & 2\tabularnewline
\textless{} 4 mm & 172\tabularnewline
\textless{} 63 mm & 15\tabularnewline
\textless{} 8 mm & 27\tabularnewline
sands & 1137\tabularnewline
silts and clays & 1140\tabularnewline
NA & 9044\tabularnewline
\bottomrule
\end{longtable}

\begin{Shaded}
\begin{Highlighting}[]
\CommentTok{#Keep only the sand fraction data (~50% of the data)}
\NormalTok{sand_harmonized  <-}\StringTok{ }\NormalTok{tss.p }\OperatorTok
\StringTok{  }\KeywordTok{filter}\NormalTok{(particle_size }\OperatorTok\StringTok{  }\KeywordTok{c}\NormalTok{(}\StringTok{'< 0.0625 mm'}\NormalTok{,}\StringTok{'sands'}\NormalTok{)) }\OperatorTok
\StringTok{  }\KeywordTok{mutate}\NormalTok{(}\DataTypeTok{conversion=}\OtherTok{NA}\NormalTok{,}
         \DataTypeTok{harmonized_parameter=}\StringTok{'p.sand'}\NormalTok{,}
         \DataTypeTok{harmonized_value=}\NormalTok{value,}
         \DataTypeTok{harmonized_unit=}\StringTok{'%'}\NormalTok{)}
\end{Highlighting}
\end{Shaded}

\hypertarget{tss-dropping-bad-units}{%
\paragraph{TSS dropping bad units}\label{tss-dropping-bad-units}}

Now that we have split out the TSS values that had ``\%'' units, we can
deal with and drop the more nonsensical or missing units. The table
below will also print out the number of ``\%'' observations that we
drop, but, remember, we kept about half of these in the above code.

Here we will convert all remaining sediment values to units of mg/L and
drop any non mass/volume units.

\begin{Shaded}
\begin{Highlighting}[]
\CommentTok{#Make a tss lookup table}
\NormalTok{tss.lookup <-}\StringTok{ }\KeywordTok{tibble}\NormalTok{(}\DataTypeTok{units=}\KeywordTok{c}\NormalTok{(}\StringTok{'mg/l'}\NormalTok{,}\StringTok{'g/l'}\NormalTok{,}\StringTok{'ug/l'}\NormalTok{,}\StringTok{'ppm'}\NormalTok{),}
                        \DataTypeTok{conversion =} \KeywordTok{c}\NormalTok{(}\DecValTok{1}\NormalTok{,}\DecValTok{1000}\NormalTok{,}\DecValTok{1}\OperatorTok{/}\DecValTok{1000}\NormalTok{,}\DecValTok{1}\NormalTok{))}


\NormalTok{tss.disharmony <-}\StringTok{ }\NormalTok{tss }\OperatorTok
\StringTok{  }\KeywordTok{anti_join}\NormalTok{(tss.lookup,}\DataTypeTok{by=}\StringTok{'units'}\NormalTok{) }\OperatorTok
\StringTok{  }\KeywordTok{filter}\NormalTok{(}\OperatorTok{!}\NormalTok{particle_size }\OperatorTok\StringTok{ }\KeywordTok{c}\NormalTok{(}\StringTok{'< 0.0625 mm'}\NormalTok{,}\StringTok{'sands'}\NormalTok{)) }\OperatorTok
\StringTok{  }\KeywordTok{group_by}\NormalTok{(units) }\OperatorTok
\StringTok{  }\KeywordTok{summarize}\NormalTok{(}\DataTypeTok{count=}\KeywordTok{n}\NormalTok{()) }



\NormalTok{knitr}\OperatorTok{::}\KeywordTok{kable}\NormalTok{(tss.disharmony,}\DataTypeTok{caption=}\StringTok{'The following TSS measurements were dropped because the units do not make sense'}\NormalTok{)}
\end{Highlighting}
\end{Shaded}

\begin{longtable}[]{@{}lr@{}}
\caption{The following TSS measurements were dropped because the units
do not make sense}\tabularnewline
\toprule
units & count\tabularnewline
\midrule
\endfirsthead
\toprule
units & count\tabularnewline
\midrule
\endhead
& 35\tabularnewline
\% & 401865\tabularnewline
count & 1\tabularnewline
kg & 29\tabularnewline
None & 16\tabularnewline
NTU & 1\tabularnewline
tons/day & 529\tabularnewline
NA & 253293\tabularnewline
\bottomrule
\end{longtable}

\hypertarget{tss-harmony-in-mgl}{%
\subsubsection{TSS harmony in mg/l}\label{tss-harmony-in-mgl}}

Now we can convert all TSS measurements to untis of `mg/l.' We do need
to do one final splitting of the data because there is another parameter
name called ``Fixed suspended solids.'' Fixed suspended solids are
essentialy the inorganic component of a sediment sample that remains
after kiln drying at 550°F. We will relable these as a harmonized
parameter `Total inorganic sediment' or tis.

\begin{Shaded}
\begin{Highlighting}[]
\CommentTok{#Join to the lookup table and harmonize units}

\NormalTok{tss.harmonized <-}\StringTok{ }\NormalTok{tss }\OperatorTok
\StringTok{  }\KeywordTok{inner_join}\NormalTok{(tss.lookup,}\DataTypeTok{by=}\StringTok{'units'}\NormalTok{) }\OperatorTok
\StringTok{  }\KeywordTok{mutate}\NormalTok{(}\DataTypeTok{harmonized_parameter =} \StringTok{'tss'}\NormalTok{,}
         \DataTypeTok{harmonized_value=}\NormalTok{value}\OperatorTok{*}\NormalTok{conversion,}
         \DataTypeTok{harmonized_unit=}\StringTok{'mg/l'}\NormalTok{) }\OperatorTok
\StringTok{  }\CommentTok{#Change harmonized parameter to tis for parameter "fixed suspended solids"}
\StringTok{  }\KeywordTok{mutate}\NormalTok{(}\DataTypeTok{harmonized_parameter =} \KeywordTok{ifelse}\NormalTok{(parameter }\OperatorTok{==}\StringTok{ 'Fixed suspended solids'}\NormalTok{,}\StringTok{'tis'}\NormalTok{,harmonized_parameter))}
\end{Highlighting}
\end{Shaded}

\hypertarget{doc}{%
\subsection{DOC}\label{doc}}

\emph{Didn't keep enough columns to really do this. Need to add
resultsampletext and a few others. Otherwise total carbon can include
fish biomass. Which is not what we are talking about}

Dissolved organic carbon is a much more complex series of parameters,
methods, and units. As with TSS we generally expect these to be in units
of mass per unit volume, but we have many more possible variations of
methods used to extract DOC values.

First let's look at the total counts for parameter unit combinations

\begin{Shaded}
\begin{Highlighting}[]
\CommentTok{#Summarize by characteristic name and unit code}
\NormalTok{doc <-}\StringTok{ }\KeywordTok{read_feather}\NormalTok{(}\StringTok{'1_wqdata/out/wqp/all_raw_doc.feather'}\NormalTok{) }\OperatorTok
\StringTok{  }\KeywordTok{wqp.renamer}\NormalTok{() }\OperatorTok
\StringTok{  }\CommentTok{#Remove trailing white space in labels}
\StringTok{  }\KeywordTok{mutate}\NormalTok{(}\DataTypeTok{units =} \KeywordTok{trimws}\NormalTok{(units)) }
\end{Highlighting}
\end{Shaded}

\begin{verbatim}
## [1] "You dropped 63589 samples because the sample medium was not labeled as Water"
\end{verbatim}

\begin{Shaded}
\begin{Highlighting}[]
\NormalTok{doc }\OperatorTok\StringTok{ }
\StringTok{  }\KeywordTok{group_by}\NormalTok{(parameter,units) }\OperatorTok
\StringTok{  }\KeywordTok{summarize}\NormalTok{(}\DataTypeTok{count=}\KeywordTok{n}\NormalTok{()) }\OperatorTok
\StringTok{  }\NormalTok{knitr}\OperatorTok{::}\KeywordTok{kable}\NormalTok{(.,}\DataTypeTok{caption=}\StringTok{'Carbon parameter names, units, and observation counts'}\NormalTok{)}
\end{Highlighting}
\end{Shaded}

\begin{longtable}[]{@{}llr@{}}
\caption{Carbon parameter names, units, and observation
counts}\tabularnewline
\toprule
parameter & units & count\tabularnewline
\midrule
\endfirsthead
\toprule
parameter & units & count\tabularnewline
\midrule
\endhead
Non-purgeable Organic Carbon (NPOC) & mg/l & 1393\tabularnewline
Organic carbon & \% & 2085\tabularnewline
Organic carbon & g/kg & 2564\tabularnewline
Organic carbon & mg/kg & 100\tabularnewline
Organic carbon & mg/l & 2009224\tabularnewline
Organic carbon & None & 510\tabularnewline
Organic carbon & ppm & 5057\tabularnewline
Organic carbon & ug/l & 627\tabularnewline
Organic carbon & NA & 25907\tabularnewline
Total carbon & mg/l & 14662\tabularnewline
\bottomrule
\end{longtable}

\hypertarget{doc-disharmony}{%
\subsubsection{DOC disharmony}\label{doc-disharmony}}

\hypertarget{doc-percent-values}{%
\paragraph{DOC percent values}\label{doc-percent-values}}

Once again we have quite a few observations of `Organic carbon' and
`Total carbon' that are in units of \% which is a perplexing unit
without some more context. Let's examine these values a little more.

\begin{Shaded}
\begin{Highlighting}[]
\NormalTok{doc.p <-}\StringTok{ }\NormalTok{doc }\OperatorTok
\StringTok{  }\KeywordTok{filter}\NormalTok{(units}\OperatorTok{==}\StringTok{'%'}\NormalTok{)}


\NormalTok{doc.min <-}\StringTok{ }\KeywordTok{read_feather}\NormalTok{(}\StringTok{'1_wqdata/out/wqp/Minnesota_doc_001.feather'}\NormalTok{) }\OperatorTok
\StringTok{  }\KeywordTok{sample_frac}\NormalTok{(.}\DecValTok{1}\NormalTok{)}

\KeywordTok{View}\NormalTok{(doc.min)}
\end{Highlighting}
\end{Shaded}

Hardest

\hypertarget{chlorophyll}{%
\subsection{Chlorophyll}\label{chlorophyll}}

\begin{Shaded}
\begin{Highlighting}[]
\CommentTok{#Read in the raw data from '1_wqdata/tmp'}
\NormalTok{chl <-}\StringTok{ }\KeywordTok{read_feather}\NormalTok{(}\StringTok{'1_wqdata/out/wqp/all_raw_chlorophyll.feather'}\NormalTok{)}
\end{Highlighting}
\end{Shaded}


\end{document}
